\documentclass{article}
\begin{document}

\begin{titlepage}
   \vspace*{\stretch{1.0}}
   \begin{center}
      \Large\textbf{Compilers: P-language}\\
      \large\textit{Sam Mylle, Federico Quin \\ University of Antwerp} \\
      \today
   \end{center}
   \vspace*{\stretch{2.0}}
\end{titlepage}

\section{Result}
We have allowed all mandatory C constructions in our grammar and we are able to create the AST and the symbol table. We have also implemented some extras, e.g. continue and break statement and for loops. \\
There is error checking. The symbol table has been adjusted to provide the reqired functionality for the translation unit. The translation unit is also fully operational.

\section{Grammar}
Currently, the grammar we wrote supports all mandatory C constructions, and of course also the optional constructions we mentioned above. The grammar can be found in the ./res directory. We have provided some comments and we tried to make the names of nonterminals as clear as possible.

\section{Sources}
Our compiler currently consisted of 5 big parts: The AST class, the ASTCreator class, the SymbolTable class, the SymbolTableBuilder class and the TypeChecker class.
First, the AST related classes make the decorated AST. Then, the symbol table related classes make the symbol table based on this AST. Finally, the TypeChecker checks whether the types of the operands match. If anything fails, it will yield an error.
Compared to the first evaluation, we now added an extra part: the translation unit (PTranslation.py). Using the resulting AST and the symboltable, the C program is translated into a P program.

\section{Testing}
For testing our compiler, we chose for pytest. We have 35 test programs, where each program tests one encapsulated part of functionality.

\end{document}